\documentclass[11pt]{article}
\usepackage[utf8]{inputenc}
\usepackage[bitstream-charter]{mathdesign}
\usepackage[T1]{fontenc}
\pagenumbering{gobble} % to remove page numbering
\usepackage{lipsum}
\usepackage[english]{babel}
\usepackage[dvipsnames]{xcolor}
\usepackage{hyperref}
\usepackage[style=numeric,sorting=none]{biblatex} %Imports biblatex package
\usepackage{geometry}
 \geometry{
 a4paper,
 total={165mm,257mm},
 top=08mm,
 bottom=10mm
}
\definecolor{awesome-skyblue}{HTML}{0395DE}

\title{\LARGE\textbf{{ITS2 cluster size and PID capability studies}}}
\author{ }
\date{ }

\begin{document}
\maketitle

The excellent pointing and momentum resolution of the ALICE Inner Tracking System (ITS2) detector makes it crucial to provide the decay-vertex position and tracking information.\\

The topology of the signal (cluster) produced by the charged particles traversing the layers is used for this purpose. The topology is encoded in the number of clusters per layer, the size of the clusters and their position.
An unforeseen abundance of large clusters in ITS2 has been observed in the data collected within the first year of data taking of Run 3, and its nature has been studied in detail. Furthermore, an inconsistency between the cluster size distribution in the data and MC simulations is observed especially for these clusters. The data-to-MC agreement can shed light on the origin of large clusters, and the latest conclusions are presented.\\

Finally, the PID capability of the ITS2 detector is discussed. Cluster topologies can be interpreted as a proxy for the energy loss by charged particles traversing the ITS2. This information, together with the reconstructed momentum of the particle and its angular distribution, can be used to assess the particle specie. The first results of the ITS2 PID capabilities study on MC simulation and data are presented for the first time.

\end{document} 